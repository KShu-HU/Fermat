% In this file you should put the actual content of the blueprint.
% It will be used both by the web and the print version.
% It should *not* include the \begin{document}
%
% If you want to split the blueprint content into several files then
% the current file can be a simple sequence of \input. Otherwise It
% can start with a \section or \chapter for instance.

\begin{theorem}[Fermat's Little theorem]
\lean{Fermatの小定理}

$p$ を素数とする。このとき $p$ と互いに素である自然数 $n$ に対して、以下の合同式が成り立つ

\[
n^{p-1} \equiv 1 \pmod{p}
\]

このことを、3 通りの方法で証明する。

1. $1$ から $p-1$ を $p$ で割った余りを用いた証明

まず、$n, 2n, \ldots, (p-1)n$ を $p$ で割った余りがどの 2 つも相異なることを示す。

任意の $i, j \in \mathbb{N}\ (0 < i \leq j < p)$ に対して $jn - in = (j - i)n$ を $p$ で割った余りが $0$ となる必要十分条件を考える。

いま、$p$ は素数かつ $n$ と互いに素であり、 $0 < j - i < p$ なので、
$(j - i)n \equiv 0 \pmod{p}$ となる場合は必ず $i = j$ である。
したがって、$n, 2n, \ldots, (p-1)n$ を $p$ で割った余りはすべて異なる。

よって、
\[
n \cdot 2n \cdot \ldots \cdot (p-1)n \equiv 1 \cdot 2 \cdot \ldots \cdot (p-1) \pmod{p}
\]

すなわち、
\[
n^{p-1}(p-1)! \equiv (p-1)! \pmod{p}
\]

$(p-1)!$ は $p$ と互いに素であるため、両辺を $(p-1)!$ で割ることで

\[
n^{p-1} \equiv 1 \pmod{p}
\]

2. 二項定理による展開を用いた帰納法による証明

\[
1^p \equiv 1 \pmod{p}
\]
は明らかである。ここで、
\[
n^p \equiv n \pmod{p}
\]
という仮定のもとで、
\[
(n+1)^p \equiv n+1 \pmod{p}
\]
を示す。

\[
(n+1)^p = \sum_{k=0}^p \binom{p}{k} n^k \cdot 1^{p-k} = \sum_{k=1}^{p-1} \binom{p}{k} n^k + n + 1
\]

ここで、$1 \leq k \leq p-1$ のとき $\binom{p}{k}$ は $p$ の倍数であるため、

\[
\sum_{k=1}^{p-1} \binom{p}{k} n^k \equiv 0 \pmod{p}
\]

ゆえに、
\[
(n+1)^p \equiv n + 1 \pmod{p}
\]

したがって、数学的帰納法により $n^p \equiv n \pmod{p}$ が成り立つ。

ここで、$n$ と $p$ が互いに素であるから、両辺を $n$ で割って、

\[
n^{p-1} \equiv 1 \pmod{p}
\]

3. ラグランジュの定理を用いた証明

ラグランジュの定理とは、有限群 $G$ とその部分群 $H$ に対して、

(1) $|G| = (G : H)|H|$

(2) 任意の $g \in G$ の位数は $|G|$ の約数である

という性質が成り立つという定理である。

これを用いてフェルマーの小定理を証明する。

有限体 $\mathbb{Z}/p\mathbb{Z}$ の乗法群 $(\mathbb{Z}/p\mathbb{Z})^{\times}$ の位数は $p - 1$ である。

ラグランジュの定理より、$n \in (\mathbb{Z}/p\mathbb{Z})^{\times}$ の位数を $d$ とすると、
これは $p - 1$ の約数なので $p - 1 = dm\ (m \in {N})$ と書ける。

したがって、

\[
n^{p-1} = n^{dm} = (n^d)^m \equiv 1^m = 1 \pmod{p}
\]

である。

\end{theorem}
